\documentclass[a4paper,11pt]{article}
\usepackage[framed,numbered]{matlab-prettifier}
\usepackage{caption}
\begin{document}
	\begin{titlepage}
		
		\newcommand{\HRule}{\rule{\linewidth}{0.5mm}} % Defines a new command for the horizontal lines, change thickness here
		
		\center % Center everything on the page
		
		
		
		\textsc{\LARGE Department of  }\\[0.3cm] % Name of your university/college
		\textsc{\LARGE Robotics and Mechatronics Engineering  }\\[0.3cm]
		\textsc{\Large   }\\[0.3cm]
		\textsc{\Large Lab report }\\[0.5cm] % Major heading such as 
		
		\HRule \\[0.4cm]
		{ \huge \bfseries DIGITAL SIGNAL PROCESSING}\\[0.4cm]  
		
		{ \huge \bfseries (CSE-401)}\\[0.03cm]
		% Title of your document
		\HRule \\[5cm]
		
		
		
		\begin{minipage}{0.4\textwidth}
			\begin{flushleft} \large
				\emph{Submitted By:}\\
				Md. Tahmeed Abdullah \\Roll: SH-092-002\\$4^{th}$ year $1^{st}$ semester % Your name
			\end{flushleft}
		\end{minipage}
		~
		\begin{minipage}{0.4\textwidth}
			\begin{flushright} \large
				\emph{Submitted To:} \\
				Mr. Sujan Sarker\\Lecturer\\Dept. of RME % Supervisor's Name
			\end{flushright}
		\end{minipage}\\[1cm]
		
	
		
		\vfill
		
		{\LARGE Date of submission: 3 February, 2019 }\\[1cm] 
		
		
		
	\end{titlepage}
	\begin{center}
		
	\end{center}
	
	\section*{Name of the experiment}
	Time shifting of digital signals.
	\section*{Objectives}
	\begin{itemize}
		\item To learn about sampling a continuous time signal and basics of digital singals. 
		\item Learn time delay and time advance of signals.
		\item Learn to implements signal processing in MATLAB
	\end{itemize}
	\section*{Theory}
	A digital signal is obtained by sampling and quantizing a continuous time signal. Let $x[n]$ be a digital signal where n indicate the sample number. Let the shifted signal is represented by $ y[n] = x[n-k] $. If $ k>0 $, the output signal $y[n]$ is delayed by samples $k$ and if $k<0$, the output signal $y[n]$ is time advanced by samples $k$. If $k=0$, the output signal is the same as input signal.
	
	For example:
	$$ x[n] = \{-x_3, -x_2, -x_1, x_0, x_1, x_2, x_3\} $$
	Here $x[n]$ is a finite digital signal and for $n=0 , x[n] = x_0$. If time advance is applied to the signal, where the signal advances by 3 samples i.e. $k=-3$. Then the output signal is as follows. 
	$$ y[n] = \{x_0, x_1, x_2, x_3, 0, 0, 0\} $$
	So, $y[n] = x[n-k]$, where $k=-3$ for this example. putting $n=0$, we get $y[0] = x[3]$ . So, the signal advances by 3 samples. 
	
	Again, if time delay is applied to the signal, where the signal is delayed by 3 samples i.e. $k=3$. Then the output signal is as follows. 
	$$ y[n] = \{0, 0, 0, -x_3, -x_2, -x_1, x_0\} $$
	So, $y[n] = x[n-k]$, where $k=3$ for this example. putting $n=0$, we get $y[0] = x[-3]$ . So, the signal delays by 3 samples. 
	
	In both the processes, some signal values get removed from the finite window and the empty spaces are filled with zero value.
		
	\vfill
	
	\section*{Implementation Code}  
	\captionsetup{labelformat=empty,labelsep=none}
	\lstinputlisting[style = Matlab-editor, caption={main.m}]{main.m}
	\vspace{1.5cm}
	\large Functions Used:
	\lstinputlisting[style = Matlab-editor, caption={delay.m}]{delay.m}
	\lstinputlisting[style = Matlab-editor, caption={advance.m}]{advance.m}
	
	
	
\end{document}